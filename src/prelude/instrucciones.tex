\section{Instrucciones}
\begin{spacing}{1.25}

Antes de estudiar o tomar en práctica este método tiene el alumno que conocer
por lo menos una parte de teoría y solfeo.

El instrumento se sostiene sentado sobre las faldas y las manos se sujetan en
las correas. quedando los pulgares afuera, pues así puede abrir o cerrar el fuelle.

La \textbf{A} que llevan los títulos abriendo y la \textbf{C} cerrando.

La palanca metálica de la mano derecha sirve para aspirar aire en caso necesario.

% The original section below used an older numbering system where 1 meant the
% index finger. That has been corrected throughout the book to match piano
% numbering where 1 means the thumb.
La \textbf{f} (forte) indica tocar con bastante fuerza ya sea al abrir o cerrar, la
\textbf{p} (piano) indica manejar con suavidad. Cada dedo tiene su número o sea la
digitación, en este caso corresponden a las do manos el igual número del mismo
dedo, pues el \textbf{2} corresponde al índice, el \textbf{3} al mayor, el \textbf{4} al
anular y el \textbf{5} al meñique. estos son los cuatro dedos de cada mano que ejecutan
sobre el teclado.

Las páginas \todo y \todo hacen recordar los signos y valores de la música, de la
página \todo hasta la \todo son para estudiar el teclado o sea identificar la
tecla a que nota corresponde, sin instrumento. Las páginas \todo y \todo son
para examinarse el conocimiento, esto debe el alumno saber con toda exactitud.
Las páginas \todo son estudios con el instrumento para ambas manos separadas.
Las páginas \todo son estudios melódicos para ambas manos. Si el alumno ha llegado
con los estudios hasta la pág. \todo, pase a los estudios técnicos pág. \todo,
y las dos partes deben seguir el mismo curso. Los estudios diarios página \todo
deben ser ejecutados a principio muy lentos aumentando la velocidad poco a poco.
El verdadero secreto es estudiar aunque sea poco pero bien y no saltear por todo
el método, sino estudiar muy bien la que está de turno para ir a la próxima, pues
quien vence todas estas dificultades presentes, dominará el teclado por completo
y será un buen maestro de su instrumento.

\begin{flushright}
EL AUTOR
\end{flushright}

\end{spacing}

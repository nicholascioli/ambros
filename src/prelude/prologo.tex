\section{Prologo}
    \footnotetext{NOTA: El indice se encuentra en la pág. \pageref{chap:indice}.}
    \label{chap:prologo}
\begin{spacing}{2}

Al presentar este método de bandoneón a los que cultiven este instrumento no quiero
menospreciar los demás métodos publicados, sino he buscado después de haber hecho
un estudio concienzudo del mismo, la forma más clara y eficaz de poder inculcar
en el alumno, mi práctica de enseñanza la que resultará por demás provechosa.
dado el acabado perfeccionamiento a que he llegado al publicar el presente
método. el cual está dividido en cuatro partes la primera son estudios teóricos
y melódicos y sirven para conocer bien la división musical, siendo estos últimos
del gran Maestro CARL CZERNY únicamente que están transportados, arreglados y
digitados para este instrumento la segunda son escalas mayores y menores con sus
correspondientes dibujados; la tercera son estudios técnicos, siendo la parte
más importante e indispensable para adquirir un buen dominio del teclado; la
cuarta son tonalidades más usuales con sus dibujos.

\begin{flushright}
EL AUTOR
\end{flushright}

\end{spacing}

\section{Teclado N°. 4 - Mano izquierda}
    \label{chap:no4}

\subsection{Notas enarmónicas abriendo}
\begin{center}
    \includesvg[width=0.8\textwidth]{./imgs/teclado_no4_iz_abrir}
\end{center}

\hspace{0.1cm} \hrule \hrule

\subsection{Notas enarmónicas cerrando}
\begin{center}
    \includesvg[width=0.8\textwidth]{./imgs/teclado_no4_iz_cerrar}
\end{center}

\lilypondfile{./no4/iz.ly}
\hspace{0.1cm} \hrule \hrule

Este teclado es para estudiar toas las notas enarmónicas.

NOTA: La palabra enarmónica quiere decir que son distintos nombres pero indican
el mismo sonido.


\pagebreak
\section{Teclado N°. 4 - Mano derecha}

\subsection{notas enarmónicas abriendo}
\begin{center}
    \includesvg[width=0.8\textwidth]{./imgs/teclado_no4_de_abrir}
\end{center}

\hspace{0.1cm} \hrule \hrule

\subsection{Notas enarmónicas cerrando}
\begin{center}
    \includesvg[width=0.8\textwidth]{./imgs/teclado_no4_de_cerrar}
\end{center}

\lilypondfile{./no4/de.ly}

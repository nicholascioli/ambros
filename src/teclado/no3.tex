\section{Teclado N°. 3 - Mano izquierda}
    \label{chap:no3}

\subsection{todo el teclado con \sharp \hspace{0.1cm} abriendo y cerrando}
\begin{center}
    \includesvg[width=0.65\textwidth]{./imgs/teclado_no3_iz_sostenido}
\end{center}

\lilypondfile{./no3/iz_sostenido.ly}
\hspace{0.1cm} \hrule \hrule

\subsection{todo el teclado con \flat \hspace{0.1cm} abriendo y cerrando}
\begin{center}
    \includesvg[width=0.65\textwidth]{./imgs/teclado_no3_iz_bemol}
\end{center}

\lilypondfile{./no3/iz_bemol.ly}
\hspace{0.1cm} \hrule \hrule

Este dibujo es para que pueda el alumno estudiar todo el teclado tanto abriendo
como cerrando.

% TODO: Use filled vs empty instead since that is more visible
NOTA: Las notas de tamaño común indican abriendo y las cuadradas cerrando.


\pagebreak
\section{Teclado N°. 3 - Mano derecha}

\subsection{todo el teclado con \sharp \hspace{0.1cm} abriendo y cerrando}
\begin{center}
    \includesvg[width=0.6\textwidth]{./imgs/teclado_no3_de_sostenido}
\end{center}

\lilypondfile{./no3/de_sostenido.ly}
\hspace{0.1cm} \hrule \hrule

\subsection{todo el teclado con \flat \hspace{0.1cm} abriendo y cerrando}
\begin{center}
    \includesvg[width=0.6\textwidth]{./imgs/teclado_no3_de_bemol}
\end{center}

\lilypondfile{./no3/de_bemol.ly}

\section{Teclado N°. 2 - Mano izquierda}
    \label{chap:no2}

\subsection{Notas con \sharp \hspace{0.1cm} y \flat \hspace{0.1cm} abriendo}
\begin{center}
    \includesvg[width=0.8\textwidth]{./imgs/teclado_no2_iz_abrir}
\end{center}

\lilypondfile{./no2/iz_abrir.ly}
\spacer

\subsection{Notas con \sharp \hspace{0.1cm} y \flat \hspace{0.1cm} cerrando}
\begin{center}
    \includesvg[width=0.8\textwidth]{./imgs/teclado_no2_iz_cerrar}
\end{center}

\lilypondfile{./no2/iz_cerrar.ly}
\spacer

Este dibujo del teclado sirve para que pueda el alumno estudiar la posición de
las notas con sostenido o bemol.


\pagebreak
\section{Teclado N°. 2 - Mano derecha}

\subsection{Notas con \sharp \hspace{0.1cm} y \flat \hspace{0.1cm} abriendo}
\begin{center}
    \includesvg[width=0.7\textwidth]{./imgs/teclado_no2_de_abrir}
\end{center}

\lilypondfile{./no2/de_abrir.ly}
\spacer

\subsection{Notas con \sharp \hspace{0.1cm} y \flat \hspace{0.1cm} cerrando}
\begin{center}
    \includesvg[width=0.7\textwidth]{./imgs/teclado_no2_de_cerrar}
\end{center}

\lilypondfile{./no2/de_cerrar.ly}
